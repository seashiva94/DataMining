\begin{homeworkProblem}[3: Objectives and Significance]
% rough draft
The objective of this project is to find trends in the dataset that
show us how the popularity of a certain business changes over
time. Further, the project aims to answer a more general question, that
does the overall sentiment of popular businesses change after being
positively or negatively reviewed by an influencer. Influencers here
can be defined as users who have are well connected on the platform,
and post reviews frequently.


This helps us understand whether there is actually an impact on how
other users perceive a business, based on the reviews of the few. If
this is the case, future popularity of a business can be predicted
based on how it has been reviewed by the influencial users, and
businesses can maximize their popularity by aimimg to obtain better
(more positive) reviews from multiple influencers.


If this is not the case however, it can be concluded based on data
that, influencers do not affect popularity in general and further
analysis must be done to find other factors that determine the general
sentiment for a business.

\end{homeworkProblem}
\clearpage

\beigin{homeworkProblem}[4: Background]
The task at hand requires appliaction of several different
techniques. Firstly, to identify trends, changes in different
attributes in the data must be observed using some time series
analysis. Time series analysis aims to find non-random patterns in
data, where successive entries inthe data are taken one after another
at a uniform time step. It also aims to forecast how these trends are
likely to change in the future with time.
% in more deatail
% http://www.statsoft.com/Textbook/Time-Series-Analysis


The project relies heavily on natural language processing, and text
mining techniques, such as keyword identification and sentiment
analysis, to analyse reviews, and score them based on the sentiment in
the text. This can be done in teaditional ways that look at words
associated with positive and negative sentiments, and score a document
based on these words. This technique however can be fooled to result in
a negitive document having a positive sentiment score, since it looks
at individual words. Deep learning has been used to train deep neural
networks, that build on top of grammatical structures and can capture
context in a document.
% in more detail both techniques
% https://nlp.stanford.edu/sentiment/


The results obtained need to be evaluated, to determine whether there
is an impact of influencers on popularity or the observations in the
data may have occured due to randomness. Hypothesis testing in
statistics is a technique that helps in determinig whether a
relationship between two datasets, defined by a hypothesis is
statistically significant.
% https://en.wikipedia.org/wiki/Statistical_hypothesis_testing


Yelp uploads their data openly as a part of the Yelp Data Challenge,
and a lot of research has been done to understand relationdships and
trends in the data.

\end{homeworkProblem}
\clearpage
% future attention
% sentiment in reviews --> rating
% game theoretic approach b/w yelp and groupon
% influencers
% clustering users based on graphs
