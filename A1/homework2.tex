%%%%%%%%%%%%%%%%%%%%%%%%%%%%%%%%%%%%%%%%%
% Programming/Coding Assignment
% LaTeX Template
%
% This template has been downloaded from:
% http://www.latextemplates.com
%
% Original author:
% Ted Pavlic (http://www.tedpavlic.com)
%
% Note:
% The \lipsum[#] commands throughout this template generate dummy text
% to fill the template out. These commands should all be removed when 
% writing assignment content.
%
% This template uses a Perl script as an example snippet of code, most other
% languages are also usable. Configure them in the "CODE INCLUSION 
% CONFIGURATION" section.
%
%%%%%%%%%%%%%%%%%%%%%%%%%%%%%%%%%%%%%%%%%

%----------------------------------------------------------------------------------------
%	PACKAGES AND OTHER DOCUMENT CONFIGURATIONS
%----------------------------------------------------------------------------------------

\documentclass{article}

\usepackage{fancyhdr} % Required for custom headers
\usepackage{lastpage} % Required to determine the last page for the footer
\usepackage{extramarks} % Required for headers and footers
\usepackage[usenames,dvipsnames]{color} % Required for custom colors
\usepackage{graphicx} % Required to insert images
\usepackage{listings} % Required for insertion of code
\usepackage{courier} % Required for the courier font
\usepackage{lipsum} % Used for inserting dummy 'Lorem ipsum' text into the template
\usepackage{url}
\usepackage{multirow}
\usepackage{booktabs}
\usepackage{amsthm}
\usepackage{amssymb}
\usepackage{amsfonts}
\usepackage{algpseudocode}
\usepackage{mathtools}

% Margins
\topmargin=-0.45in
\evensidemargin=0in
\oddsidemargin=0in
\textwidth=6.5in
\textheight=9.0in
\headsep=0.25in

\linespread{1.1} % Line spacing

% Set up the header and footer
\pagestyle{fancy}
\lhead{\hmwkAuthorName} % Top left header
\chead{\hmwkClass\ (\hmwkClassInstructor\ \hmwkClassTime): \hmwkTitle} % Top center head
\rhead{\firstxmark} % Top right header
\lfoot{\lastxmark} % Bottom left footer
\cfoot{} % Bottom center footer
\rfoot{Page\ \thepage\ of\ \protect\pageref{LastPage}} % Bottom right footer
\renewcommand\headrulewidth{0.4pt} % Size of the header rule
\renewcommand\footrulewidth{0.4pt} % Size of the footer rule

\setlength\parindent{0pt} % Removes all indentation from paragraphs

%----------------------------------------------------------------------------------------
%	CODE INCLUSION CONFIGURATION
%----------------------------------------------------------------------------------------

\definecolor{MyDarkGreen}{rgb}{0.0,0.4,0.0} % This is the color used for comments
\lstloadlanguages{R} % Load Perl syntax for listings, for a list of other languages supported see: ftp://ftp.tex.ac.uk/tex-archive/macros/latex/contrib/listings/listings.pdf
\lstset{language=R, % Use Perl in this example
        frame=single, % Single frame around code
        basicstyle=\small\ttfamily, % Use small true type font
        keywordstyle=[1]\color{Blue}\bf, % Perl functions bold and blue
        keywordstyle=[2]\color{Purple}, % Perl function arguments purple
        keywordstyle=[3]\color{Blue}\underbar, % Custom functions underlined and blue
        identifierstyle=, % Nothing special about identifiers                                         
        commentstyle=\usefont{T1}{pcr}{m}{sl}\color{MyDarkGreen}\small, % Comments small dark green courier font
        stringstyle=\color{Purple}, % Strings are purple
        showstringspaces=false, % Don't put marks in string spaces
        tabsize=5, % 5 spaces per tab
        %
        % Put standard Perl functions not included in the default language here
        morekeywords={},
        %
        % Put Perl function parameters here
        morekeywords=[2]{on, off, interp},
        %
        % Put user defined functions here
        morekeywords=[3]{test},
       	%
        morecomment=[l][\color{Blue}]{...}, % Line continuation (...) like blue comment
        numbers=left, % Line numbers on left
        firstnumber=1, % Line numbers start with line 1
        numberstyle=\tiny\color{Blue}, % Line numbers are blue and small
        stepnumber=5 % Line numbers go in steps of 5
}

% Creates a new command to include a perl script, the first parameter is the filename of the script (without .pl), the second parameter is the caption
\newcommand{\rscript}[2]{
\begin{itemize}
\item[]\lstinputlisting[caption=#2,label=#1]{#1.R}
\end{itemize}
}

%----------------------------------------------------------------------------------------
%	DOCUMENT STRUCTURE COMMANDS
%	Skip this unless you know what you're doing
%----------------------------------------------------------------------------------------

% Header and footer for when a page split occurs within a problem environment
\newcommand{\enterProblemHeader}[1]{
\nobreak\extramarks{#1}{#1 continued on next page\ldots}\nobreak
\nobreak\extramarks{#1 (continued)}{#1 continued on next page\ldots}\nobreak
}

% Header and footer for when a page split occurs between problem environments
\newcommand{\exitProblemHeader}[1]{
\nobreak\extramarks{#1 (continued)}{#1 continued on next page\ldots}\nobreak
\nobreak\extramarks{#1}{}\nobreak
}

\setcounter{secnumdepth}{0} % Removes default section numbers
\newcounter{homeworkProblemCounter} % Creates a counter to keep track of the number of problems

\newcommand{\homeworkProblemName}{}
\newenvironment{homeworkProblem}[1][Problem \arabic{homeworkProblemCounter}]{ % Makes a new environment called homeworkProblem which takes 1 argument (custom name) but the default is "Problem #"
\stepcounter{homeworkProblemCounter} % Increase counter for number of problems
\renewcommand{\homeworkProblemName}{#1} % Assign \homeworkProblemName the name of the problem
\section{\homeworkProblemName} % Make a section in the document with the custom problem count
\enterProblemHeader{\homeworkProblemName} % Header and footer within the environment
}{
\exitProblemHeader{\homeworkProblemName} % Header and footer after the environment
}

\newcommand{\problemAnswer}[1]{ % Defines the problem answer command with the content as the only argument
\noindent\framebox[\columnwidth][c]{\begin{minipage}{0.98\columnwidth}#1\end{minipage}} % Makes the box around the problem answer and puts the content inside
}

\newcommand{\homeworkSectionName}{}
\newenvironment{homeworkSection}[1]{ % New environment for sections within homework problems, takes 1 argument - the name of the section
\renewcommand{\homeworkSectionName}{#1} % Assign \homeworkSectionName to the name of the section from the environment argument
\subsection{\homeworkSectionName} % Make a subsection with the custom name of the subsection
\enterProblemHeader{\homeworkProblemName\ [\homeworkSectionName]} % Header and footer within the environment
}{
\enterProblemHeader{\homeworkProblemName} % Header and footer after the environment
}

%----------------------------------------------------------------------------------------
%	NAME AND CLASS SECTION
%----------------------------------------------------------------------------------------

\newcommand{\hmwkTitle}{Homework\ \#2} % Assignment title
\newcommand{\hmwkDueDate}{Thursday, September 14, 10:00P} % Due date
\newcommand{\hmwkClass}{B565 Fall 2017} % Course/class
\newcommand{\hmwkClassTime}{TR 1115-1330} % Class/lecture time
\newcommand{\hmwkClassInstructor}{Dr. Dalkilic} % Teacher/lecturer
\newcommand{\hmwkAuthorName}{Your Name} % Your name

%----------------------------------------------------------------------------------------
%	TITLE PAGE
%----------------------------------------------------------------------------------------

\title{
\vspace{2in}
\textmd{\textbf{\hmwkClass:\ \hmwkTitle}}\\
\normalsize\vspace{0.1in}\small{Due\ on\ \hmwkDueDate}\\
\vspace{0.1in}\large{\textit{\hmwkClassInstructor\ }}
\vspace{3in}
}

\author{\textbf{\hmwkAuthorName}}
\date{\today} % Insert date here if you want it to appear below your name

%----------------------------------------------------------------------------------------

\begin{document}

\maketitle

%----------------------------------------------------------------------------------------
%	TABLE OF CONTENTS
%----------------------------------------------------------------------------------------

%\setcounter{tocdepth}{1} % Uncomment this line if you don't want subsections listed in the ToC

\newpage
\tableofcontents
\newpage

%----------------------------------------------------------------------------------------
%	PROBLEM 1
%----------------------------------------------------------------------------------------

% To have just one problem per page, simply put a \clearpage after each problem

\begin{homeworkProblem}[Problem 1: Becoming Familiar with Data]
 This problem examines Wolberg's breast cancer data\cite{WMbreast90}.  This set, though tiny, provides a good start for us.  The site is:\\
\small {\url{http://archive.ics.uci.edu/ml/machine-learning-databases/breast-cancer-wisconsin/}}
Please red the data description to more fully understand the format. We establish some attribute names to make discussion easier.  We let $\mathbb{N} = \{0,1,2,\ldots\}$.


\begin{table}[h] 
\centering
\begin{tabular}{*{4}{rrrr}}
\toprule
\textsf{ID} & \textsf{Description} & \textsf{Domain} & \textsf{Attribute Name} \\ 
\midrule
 1. & Sample code number        &    string & SCN \\ 
   2.& Clump Thickness             &  $\mathbb{N}$ & $A_2$ \\
   3.& Uniformity of Cell Size     &  $\mathbb{N}$ & $A_3$\\
   4.& Uniformity of Cell Shape  &   $\mathbb{N}$ & $A_4$\\
   5.& Marginal Adhesion           &  $\mathbb{N}$ & $A_5$\\
   6.& Single Epithelial Cell Size  & $\mathbb{N}$ & $A_6$\\
   7.& Bare Nuclei                  & $\mathbb{N}$ & $A_7$\\
   8.& Bland Chromatin          &     $\mathbb{N}$ & $A_8$\\
   9.& Normal Nucleoli           &    $\mathbb{N}$ & $A_9$\\
  10.& Mitoses                       & $\mathbb{N}$ & $A_{10}$\\
  11.& Class:                       & char &  $C$ \\ 
\bottomrule
  \end{tabular}
\end{table}

Familiarize yourself with this data set.
\end{homeworkProblem}

\begin{homeworkProblem}[Problem 2: Statement, Consequences, and Pre-processing]
 Suppose you're working to help a clinic serve a community that has limited resources to identify and treat breast cancer.  The cost of a biopsy is from \$1000 to \$5000, since it requires a pathologist.  The cost of a masectomy is \$15,000 to \$55,000 (these are representative costs in 2017).   

 \begin{enumerate} \item What was the total cost (lower and upper bounds) of the biopsies in $\Delta$ when done by a pathologist?   \item What would have been the likely total cost of masectomies for the malignant tumors? \item Assuming a 70\% mortality rate for untreated in year five (and that all women were untreated), how many deaths does the data suggest in five years? \item Compose a succint problem statement that you imagine is pertinent to this scenario.  \item Ignoring the SCN and $C$ columns, how many attributes (or features) does $\Delta$ have? \item Let $\Delta_{u}\subset \Delta$ be the data that has missing values.  How many missing values exist (total)?  What is the size of $\Delta_u$? \item How many patients have missing values? \item Give the SCNs for any patient in $\Delta_u$.  \item Of these data, would you have recommended re-examination for the women?  Revise your total cost (lower and upper bounds) of the biopsies.
  \item Is the amount of missing data significant from an algorithmic perspective? \item Assess the \textit{human} significance of either keeping or removing the tuples with unknown data. You should consider the human element too.
 \end{enumerate} 
 
\end{homeworkProblem}



\begin{homeworkProblem}[Problem 3: Setting-up Infrastructure and Visualization]
 Using either \textsf{MySQL}, \textsf{SQL Server} or \textsf{PostgreSQL}, build a table and load the data set.  Connect to \textsf{R} so that you can quickly and easily perform analysis.  Using \texttt{R}, plot histograms for each attribute and $C$. In other words, write a loop in \textsf{R} that queries the database and plots a histogram for each attribute.  Add the plots to the homework file.

 


We can use \textsf{R} to easily   Here is the beginning of an R session:
\begin{verbatim}
> install.packages("data.table")
> library(data.table)
> install.packages("curl")
> mydata <- fread("https://archive.ics.uci.edu/ml/machine-learning-databases/
                   breast-cancer-wisconsin/breast-cancer-wisconsin.data")
> head(mydata)
        V1 V2 V3 V4 V5 V6 V7 V8 V9 V10 V11
1: 1000025  5  1  1  1  2  1  3  1   1   2
2: 1002945  5  4  4  5  7 10  3  2   1   2
3: 1015425  3  1  1  1  2  2  3  1   1   2
4: 1016277  6  8  8  1  3  4  3  7   1   2
5: 1017023  4  1  1  3  2  1  3  1   1   2
6: 1017122  8 10 10  8  7 10  9  7   1   4
> 
\end{verbatim}
\end{homeworkProblem}

\begin{homeworkProblem}[Problem 4: Partitions]
Let $X \subset \Sigma^*$ be a finite set (of finite length) strings over the alphabet $\Sigma$. Demonstrate whether the relations are equivalence relations.  We write $length(x)$ to mean the size of the word. For some $x \in X$ we write $x[i]$ to mean the $i^{th}$ character $0 \leq i \leq length(x) - 1$.
\begin{itemize}
\item $R_1 \subset X^2$ such that $(x,y) \in R$ when $length(x) = length(y)$.
\item $R_2 \subset X^2$ such that $(x,y) \in R$ when $x[0] = y[0]$.
\item $R_3 \subset X^2$ such that $(x,y) \in R$ if $x,y$ share at least one character.
\end{itemize}
\end{homeworkProblem}
%----------------------------------------------------------------------------------------
\bibliographystyle{unsrt} 
\bibliography{hw2}
\end{document}